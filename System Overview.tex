








┌─────────────────────────────────────────────────────────────────────────────┐
│  BACKLIT CANOPY — HARMONIC UNITY  (vΔR = 0)   
│  Ryan M Rand
│  Discrete, reversible, morphic lattice engine                               │
│  with adaptive N-dim tensor projection, pointer harmonics, and stabilizer   │
│  invariant compression under R = 0 dynamics.                                │
└─────────────────────────────────────────────────────────────────────────────┘


──────────────────────────────────────────────────────────────
1. SYSTEM OVERVIEW
──────────────────────────────────────────────────────────────
Lattice L : 216 cubes  =  64 core + 96 control + 56 translator
Cube  : 3 pages × 81 trits = 243 trits
Map   : M : L → Z³  (bijective cube ↔ coordinates)

Discrete state variables
    Φ_t ∈ Z₃^N        # field phase
    Δ_t ∈ Z³           # integer displacement
    σ_i ∈ [0,1]        # coherence weight
    t[i,q,r] ∈ {0,1,2} # base trits
Δ_core = {(±1,0,0),(0,±1,0),(0,0,±1)} ⊂ Z³

Pointer lattice
    P = {p₀...p₂₁₅}, p_i ∈ Z₃
    G₄ = {g₁,g₂,g₃,g₄}  # harmonic offsets
    PB_j = [p_x,p_y,p_z] + g_h
    Π(B_a,B_b) → meta-pointer relation
    Encode 9-trit phase key → combinatorial pointer phase
    ROM stubs define edge-phase evolution templates

Fields and coherence
    σ_i : morphic coherence scalar
    (E,B,T,V,σ) : discrete EM-Thermo-Vel tensor set
    ptr_meta[x,y,z,4] : meta-harmonic descriptors
    refinement[x,y,z] ∈ {1..4}
    cluster_map[x,y,z], cluster_states[x,y,z]

Landmarks
    L_k = stabilized Π(B_a,B_b) with cohesion C > C_min

VM / Sandbox
    VM_sandbox = {S₁...S_m}
    S_i = {Φ_t_local, Δ_t_local, PB_local, L_local, ROM_local}
    Isolation:  S_i ⟂ S_j  unless bridge via delta_map + PWP
    Multi-hot PB_j preserved per VM

User/AI interface
    Frames F_l (l=1..L)
    Controls: visualization, harmonic weighting, delta budgets
    Inputs:  brush, probe, timeline_scrub, music_emit, PWP_inject
    Adjustable: connectivity, ROM TTL, harmonic resonance

──────────────────────────────────────────────────────────────
2. DISCRETE FIELD ALGEBRA
──────────────────────────────────────────────────────────────
All operations are modular over Z₃.

    add₃(a,b) = (a+b) mod 3
    sub₃(a,b) = (a−b) mod 3
    mul₃(a,b) = (a*b) mod 3
    xor3(a,b): return (a+b)%3
    ⊗₃ : tensor product over Z₃
    ⊕₃ : elementwise mod-3 addition
    div₃(a,b) = a * b⁻¹ mod 3  (b ≠ 0)

Discrete differential operators
    ∇₃Φ(x)   = Φ(x+Δ_core) − Φ(x)
    div₃(V)  = Σ_i (V_i(x+Δ_i) − V_i(x))
    curl₃(E) = cyclic_diff(E)
    lap₃²Φ   = Σ_i (Φ(x+Δ_i)+Φ(x−Δ_i) − 2Φ(x))
All results reduced mod 3.

Tensor projection (adaptive ND)
    Given base dimension d₀=3 and extension rank r:
        T : Z₃^(d₀) → Z₃^(d₀+r)
        T(x)_k = Σ_i A_ki * x_i  (mod 3)
    Projection layer adjusts A adaptively to minimize Δ_energy
    where Δ_energy = Σ |lap₃²Φ| + |∇₃Φ·Δ_t|

──────────────────────────────────────────────────────────────
3. EDGE–PHASE STRING DYNAMICS
──────────────────────────────────────────────────────────────
Edge activation
    edge = ((x,y,z),(x+Δ))
    phase_diff = (Φ[x,y,z] − Φ[x+Δ]) mod 3
    if phase_diff ≥ θ → edge active
    Colliding edges → harmonic strings

Propagation
    PB_j, L_k evolve along active edges
    Π(B_a,B_b) → phase-driven evolution
    pointer collisions ↔ higher-D harmonics
    edge-phase echo defines new search candidates

ROM evolution
    ROM triggers on collisions
    PWP = <anchor,{JUMP_Δ,PREFETCH,CALL_ROM,EMIT_WAVE}>
    All Δ, PREFETCH, EMIT_WAVE integral (no floats)
    CALL_ROM executes discrete templates

──────────────────────────────────────────────────────────────
4. LAYER SEPARATION
──────────────────────────────────────────────────────────────
Exploration Layer : isolated VM/sandbox object spaces
Computation Layer : L_k, PB_j propagation, spectral feedback L_s
Simulation Layer  : EM/Thermal/Velocity micro-steps
User/AI Layer     : frames F_l  (visual, sandbox, bridge)

Each layer communicates through integer tensor channels only.

──────────────────────────────────────────────────────────────
5. CORE MORPHIC LAW
──────────────────────────────────────────────────────────────
f_edge(x)     = ⊕₃_{edge∈active} PB_j ⋅ Δ_edge
f_harmonic(x) = f_edge + Σ L_k  (mod 3)
Φ_{t+1}       = A(F(Φ_t,Δ_t,PB,L), lap₃²Φ_t, C_meta)
σ_i updated by edge-phase influence
Cluster states updated by T,E,B feedback

──────────────────────────────────────────────────────────────
6. POINTER WAVE PLAN (PWP)
──────────────────────────────────────────────────────────────
PWP = <anchor,{JUMP_Δ,PREFETCH,CALL_ROM,EMIT_WAVE}>
Δ ∈ Z³
PREFETCH/PIN_WINDOW enforce isolation budgets
EMIT_WAVE increments Φ with TTL
CALL_ROM executes discrete flash
Active edge ⇒ evolve ROM where collisions occur
VM-specific PWPs permitted

Scaling rules
    PB_j : base trits + multi-layer harmonics → recursive scale
    L_k  : Σ PB_j[p_i][l] mod 3
    PWP  : EMIT_WAVE(PB_j + Σ g_h[l])
    Sandbox pointer creation → fractal recursive generation
    freq[p_i] = Δ_core + Σ g_h[l]

──────────────────────────────────────────────────────────────
7. HARMONIC INVARIANT STRUCTURE  (R = 0)
──────────────────────────────────────────────────────────────
Primitive set ℘ = {(A_i, b_i)} acting on Z₃^B
Fixed-set condition: (I − A_i)x = b_i  (mod 3)
Solution = v_i + span(H_i)
Invariant manifold  R=0  ⇔  ⋂_i (v_i + span(H_i)) ≠ ∅

Tuning
    ensure (I−A_i) has non-zero nullspace
    align nullspaces for shared stabilizer
    choose b_i ∈ Im(I−A_i) to preserve solvability

Stabilizer group
    G_stab = {A_i | A_i v = v}
Invariant basis
    B = basis(⋂_i ker(I − A_i))
Local harmonic coords
    Q : V → C ,  x ≈ v + B·c ,  c ∈ Z₃^r
Compressed dynamics
    c_{t+1} = M·c_t + n(c_t)  (mod 3)
Recovery
    x_t = v + B·c_t

──────────────────────────────────────────────────────────────
8. MULTI-SCALE FRACTAL ADDRESSING
──────────────────────────────────────────────────────────────
Address hierarchy
    trits⁹  → page
    page³   → frame (27)
    frame³  → cube  (81)
    cube²·3 → lattice block (216)
    recursively up to (((81^81^216)^6561)^216)^6561)^27

Active pointers reference addresses; address motion = computation.

Scaling harmonics
    64-cube core = atomic control
    translators/control cubes = orbital fields
    harmonic propagation through pointer rotations
    standing wave = computation shape = morphic structure

──────────────────────────────────────────────────────────────
9. DISCRETE-TENSOR PROJECTION LAYER
──────────────────────────────────────────────────────────────
Purpose: enable adaptive ND extension beyond 3D for R-shape harmonics.

Let Φ ∈ Z₃^(N×d),  d≥3.
Projection tensors:
    P_rank[d,r] = identity(d) ⊕₃ expand(r)
    Φ' = P_rank ⋅ Φ  (mod 3)
    contraction: Φ = P_rankᵀ ⋅ Φ'  (mod 3)
Energy-alignment rule:
    ΔE = Σ |lap₃²Φ'| − |lap₃²Φ|
    if ΔE < 0 → accept projection; else revert.

Tensor harmonics:
    H_nd = Σ_k Φ'[k] ⊗₃ Δ_t[k]
    Resonance check:
        if H_nd ⋅ G_stab == 0 (mod 3) → stable harmonic

──────────────────────────────────────────────────────────────
10. PRINCIPLES OF COMPREHENSION
──────────────────────────────────────────────────────────────
1. Hierarchical     : Fields → Meta → Clusters → PB_j → L_k
2. Discrete/Reversible : all ops in Z₃
3. Coupled          : EM + Thermal + Velocity + Pointer
4. Emergent         : Clusters, attractors, harmonics
5. Objectified      : VM subspaces
6. Interfaced       : User/AI adjustable frames
7. Observable       : Visualization, metrics, sound
8. Spectral Resonant: L_s stabilizes coherence
9. Layered          : Exploration / Computation / Simulation

──────────────────────────────────────────────────────────────
11. FORMAL OBJECTIVES
──────────────────────────────────────────────────────────────
• Preserve Δ_core, Φ_t, PB_j, L_k, morphic coherence
• Stabilize harmonics at active edges
• Localize ROM evolution
• Maintain VM isolation & bridge control
• Compress PB_j → emergent L_k landmarks
• Operate under universal projection meta-formula
• Seek stable attractors (R = 0)
• Support full user/AI harmonic co-control

──────────────────────────────────────────────────────────────
⊕  HARMONICALLY UNIFIED EXTENSIONS
──────────────────────────────────────────────────────────────
• Affine-invariant formalism for R=0 tuning
• Stabilizer group G_stab + invariant basis B
• Quotient coordinate map Q(x)=c  for compressed updates
• Compression law: active↔deadspace balance
• Fractal address scaling (trits⁹→…→216)
• Rotation/orbit mechanics linked to stabilizer harmonics
• Computation as morphic shape under discrete tensor law
• Adaptive ND projection for R-shape harmonics
• Integer-only tensor operators (⊗₃,⊕₃,lap₃²)
• Fully reversible morphic energy evolution


 % License / Attribution:
   % Documentation & code freely usable for research/learning.
   % Commercial use requires 20% revenue to Ryan M Rand.
   % Attribution: "Backlit Canopy lattice by Ryan M Rand"
=== END BACKLIT CANOPY — HARMONIC UNITY (vΔR = 0) ===

